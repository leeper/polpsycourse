\documentclass[12pt,a4paper]{article}
\usepackage[top=1in, bottom=1in, left=1in, right=1in]{geometry}
\usepackage{graphicx,setspace,hyperref,mdwlist,comment}
% mini table of contents
\usepackage{minitoc}
\dosecttoc % make section toc
\setcounter{secttocdepth}{2} % subsection depth
\renewcommand{\stctitle}{} % no title
\nostcpagenumbers

\setlength{\marginparwidth}{.5in}
\usepackage{natbib}
% Two lines to create in-text full citations for a syllabus
% And comment out my other standard bibtex commands
\usepackage{bibentry}
\newcommand{\reading}[2][]{\noindent --{#1} \bibentry{#2}.\vspace{.25em}\\}
\newcommand{\seealso}{\noindent \emph{See Also:}\\}
\newcommand{\topic}[1]{\noindent \textbf{#1}\\}
\usepackage[T1]{fontenc}
\usepackage{lmodern}
\hypersetup{
    bookmarks=true,         % show bookmarks bar?
    unicode=false,          % non-Latin characters in Acrobat’s bookmarks
    pdftoolbar=true,        % show Acrobat’s toolbar?
    pdfmenubar=true,        % show Acrobat’s menu?
    pdffitwindow=false,     % window fit to page when opened
    pdfstartview={FitH},    % fits the width of the page to the window
    pdftitle={Syllabus: Advanced Political Psychology Seminar},    % title
    pdfauthor={Thomas J. Leeper},     % author
    pdfsubject={Political Science},   % subject of the document
    pdfnewwindow=true,      % links in new window
    pdfborder={0 0 0}
}

\title{Advanced Political Psychology Seminar}
\author{Thomas J. Leeper\\
Department of Political Science and Government\\
Aarhus University}

\begin{document}
\nobibliography*

\maketitle

\faketableofcontents

%\section{Introduction}

\section{Objectives}
The learning objectives for the course are as follows. By the end of the course, students should be able to:

\begin{enumerate}
\item Objectives Here
\end{enumerate}

\section{Exam}
Students will be evaluated via a home essay that states an original research question, introduces theory to address that question, and designs an appropriate empirical test to assess theorized expectations.

\section{Reading Material}
The assigned material for the course consists of empirical research articles, all of which are available online.

\clearpage
\section{Schedule}
The general schedule for the course is as follows. Details on the readings for each week are provided on the following pages.

\secttoc

\clearpage


\clearpage
\subsection{Week 1}
\emph{Topic}
\vspace{1em}

\subsubsection*{Readings}

\seealso



% load bibtex, but don't generate a bibliography
\bibliographystyle{plain}
\nobibliography{Syllabus}

\end{document}
